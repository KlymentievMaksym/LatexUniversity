\documentclass{article}
\usepackage[english, ukrainian]{babel}
\usepackage[fontsize = 14pt]{fontsize}
\usepackage{fontspec}
\setsansfont{CMU Sans Serif}
\setmainfont{CMU Serif}
\setmonofont{CMU Typewriter Text}
\usepackage{microtype}

\usepackage{hyperref}
\hypersetup{colorlinks=true, linkcolor=[RGB]{255, 3, 209}, citecolor={black}}

\usepackage{amsmath}
% \usepackage{amsthm}
% \usepackage{amssymb}

\usepackage{graphicx}

% \usepackage{caption}

\usepackage{wrapstuff}
\usepackage[ragged]{sidecap}

\usepackage[
    a4paper,
    footskip=1cm,
    headsep=0.3cm,
    top=2cm,
    bottom=2cm,
    left=2cm,
    right=2cm,
    showframe
]{geometry}
% \geometry{left=25mm,right=15mm,top=20mm,bottom=20mm}

\begin{document}

    \newtheorem{lemma}{Лема}

    % TODO: Do chapter or smth like these
    \begin{center}
        Рівневий набір гармонійних функцій

        А. В. Тор
    \end{center}

    % \frac{\pi}{2}
    Для $ \theta \in [ 0, \pi/2 [ $, розглянемо множини
    % TODO:
    \begin{equation*}
        ...
    \end{equation*}
    % TODO:
    \begin{equation*}
        ...
    \end{equation*}
    % TODO:
    \begin{equation*}
        ...
    \end{equation*}
    %
    де $ p_a(z) $ --- комплексний многочлен, визначений формулою
    % TODO:
    \begin{equation*}
        ...
    \end{equation*}

    \begin{lemma}
        Нехай $ \theta \in [0, \frac{\pi}{2}] $.
        Тоді кожна з множин $ \Sigma_{1, \theta} $ та $ \Sigma_{-1, \theta} $ утворюється двома гладкими кривими, які локально ортогональні відповідно при $ z = 1 $ та $ z = -1 $ точніше:
        %
        \begin{equation*}
            ...
        \end{equation*}
        % TODO:
        \begin{equation*}
            ...
        \end{equation*}
        %
        Дві криві, що визначають $ \Sigma_{1, \theta} $ (відповідно $ \Sigma_{-1, \theta} $), перетинаються лише при $ z = 1 $ (відповідно $ z = -1 $).
        Більше того, для $ \theta \notin \{ 0, \frac{\pi}{2} \} $, вони розходяться по-різному до $ \infty $ в одному з напрямків
        % TODO:
        \begin{equation*}
            ...
        \end{equation*}
        %
        Для $ \theta = 0 $, (відповідно $\theta = \frac{\pi}{2} $), один промінь $ \Sigma_{1, \theta} $ (відповідно $ \Sigma_{-1, \theta} $) розходиться до $ z = -1 $ (відповідно $ z = 1 $).
    \end{lemma}

    \begin{proof}
        Нехай задано непостійну гармонічну функцію $ u $, визначену в деякій області $ D $ of $ \mathbb{C}  $.
        Критичними точками $ u $ є саме ті, де
        % TODO:
        \begin{equation*}
            ...
        \end{equation*}
        %
        Вони ізольовані.
        Якщо $ v $ є гармонічним спряженням $ u $ у $ D $, скажімо, $ f (z) = u (z) + iv (z) $ аналітична у $ D $, тоді за Коші-Ріманом,
        % TODO:
        \begin{equation*}
            ...
        \end{equation*}
        %
        Встановлений рівень
        % TODO:
        \begin{equation*}
            ...
        \end{equation*}
        %
        $u$ через точку $ z_0 \in D $ залежить від поведінки $f$ поблизу $z_0$. Точніше, якщо $z_0$ є критичною точкою $u$, ( $ u′ (z_0) = 0 $ ), то існує околиця $\mathcal{U}$ околу $z_0$, голоморфної функції $g (z)$ визначена на $\mathcal{U}$, така, що
        % TODO:
        \begin{equation*}
            ...
        \end{equation*}
        %
        Взявши гілку $m$-го кореня з $g (z)$ , $f$ має локальну структуру
        % TODO:
        \begin{equation*}
            ...
        \end{equation*}
        %
        Звідси випливає, що $ \Sigma_{z_0} $ локально утворена $m$ аналітичними дугами які проходять через $z_0$ і перетинаються там під рівними кутами $\pi/m$.
        Через регулярну точку $z_0 \in D $, ( $ u \prime (z_0) \neq 0 $ ), теорема про неявну функцію стверджує, що $ \Sigma_{z_0} $ є локально єдиною аналітичною дугою.
        Зауважте, що множина рівнів гармонічної функції не може закінчуватися у звичайній точці.

        Розглянемо багатозначну функцію
        % TODO:
        \begin{equation*}
            ...
        \end{equation*}
        %
        Інтегруючи вздовж відрізка [1, a] , можна припустити, що без втрати загальності, що
        %
        \begin{equation}
            \label{eq:1}
            ...
        \end{equation}
        %
        Очевидно, що:
        % TODO:
        \begin{equation*}
            ...
        \end{equation*}
        %
        Отже, при фіксованому виборі аргументу та квадратного кореня всередині інтеграла, $f1,\theta$ та $g$ є однозначними аналітичними функціями в $\mathbb{C} \setminus ] -\infty, -1]$.

        Припустимо, що для деяких $a \in \mathbb{C} \setminus ] -\infty, -1], a \ne 1 $,
        % TODO:
        \begin{equation*}
            ...
        \end{equation*}
        %
        Тоді,
        % TODO:
        \begin{equation*}
            ...
        \end{equation*}
        %
        Беручи справжні деталі, ми отримуємо
        % TODO:
        \begin{equation*}
            ...
        \end{equation*}
        %
        За неперервністю функцій всередині цих інтегралів на відрізку $ [0, 1] $, існують $ t1, t2 \in [0, 1] $ такі що
        % TODO:
        \begin{equation*}
            ...
        \end{equation*}
        %
        а потім
        % TODO:
        \begin{equation*}
            ...
        \end{equation*}
        %
        Взявши їх співвідношення, отримуємо
        % TODO:
        \begin{equation*}
            ...
        \end{equation*}
        %
        яка не може виконуватись, оскільки, якщо $ \Im a > 0 $, то
        % TODO:
        \begin{equation*}
            ...
        \end{equation*}
        %
        Випадок $\Im a < 0$ є аналогічним, тоді як випадок $a \in R$ можна легко відкинути.
        Таким чином, $ a = 1 $ є єдиною критичною точкою $\Re f_{1,\theta}$.
        Since $ f \prime{} \prime{}_{1,\theta}(1) = 2g (1)̸ \ne 0 $, виводимо локальну поведінку $\Sigma_{1, \theta}$ поблизу $ a = 1 $.

        % TODO: BREAK
        Припустимо, що для деяких $ \theta \in ]0, \frac{\pi}{2} [ $, промінь $\Sigma_{\pm1, \theta}$ розходиться до певного моменту в $]−∞, −1[$; або приклад,
        % TODO:
        \begin{equation*}
            ...
        \end{equation*}
        %  TODO: ADD formulas
        Нехай $ \epsilon > 0 $ так, що $ 0 < \theta - 2 \epsilon $. Для $ a \in \mathbb{C} $ задовольняє $ \pi - \epsilon < \arg{a} < \pi $, $ 0 < \theta - 2 \epsilon < \theta + 2 \arg{a} + arg{\int_{0}^{1}{\sqrt{t(1-t)}\sqrt{t(a-1)+2}dt}} < \frac{\pi}{2} + \theta - \frac{\epsilon}{2} < \pi $, що суперечить \eqref{eq:1}.
        Інші випадки подібні.
        Таким чином, будь-який промінь з $\Sigma_{\pm1, \theta}$ повинен розходитись на $\infty$.
        Випадок $\theta = 0$ є простішим.

        Якщо $a \to \infty$, тоді $ \left\lvert f_{1,\theta}(a) \right\rvert \to +\infty $; since $\Re f_{1, \theta}(a) = 0$, we have $\left\lvert \Im f_{1,\theta}(a) \right\rvert \to +\infty$. Звідси випливає, що
        % TODO:
        \begin{equation*}
            ...
        \end{equation*}
        Ми отримуємо поведінку будь-якої дуги $\Sigma_{1, \theta}$, яка розходиться до $\infty$.
        Зокрема, з принципу максимуму модуля, два промені з $\Sigma_{, \theta}$ не можуть розходитись у $\infty$.
        $\Sigma_{1, \theta}$ не можуть розходитись до $\infty$ в одному напрямку.

        Якщо $\Sigma_{1, \theta}$ містить регулярну точку $z_0$ (наприклад, $\Im z_0 > 0$), яка не належить дугам $\Sigma_{1, \theta}$, що виходять з точки $a = 1$.
        Два промені кривої набору рівнів $\gamma$, що проходять через $z_0$ розходяться до $\infty$ у двох різних напрямках.
        Звідси випливає, що γ має проходити через $z_1 = 1 + iy$, для деяких $y > 0$, або $z_1 = y$, для деяких $y > 1$.
        Легко перевірити, що в обох випадках, для будь-якого вибору аргументу,
        % TODO:
        \begin{equation*}
            ...
        \end{equation*}
        і отримуємо протиріччя.
        Таким чином, $\Sigma_{1, \theta}$ утворюється лише двома двома кривими, що проходять через $a = 1$.
        Таку саму ідею дає структура $\Sigma_{-1, \theta}$; навіть більше, з співвідношення
        % TODO:
        \begin{equation}
            \label{eq:2}
            ...
        \end{equation}
        доступних для довільного $ \theta \in [\pi/4, \pi/2[$ , можна легко побачити що $\Sigma_{-1, \frac{\pi}{2} - \theta}$ і $\Sigma_{1, \theta}$ симетричні відносно уявної осі \eqref{eq:2}.
        Це приводить нас до того, щоб обмежити наше дослідження випадком.
    \end{proof}

\end{document}