\documentclass{article}
\usepackage[english, ukrainian]{babel}
\usepackage[fontsize = 14pt]{fontsize}
\usepackage{fontspec}
\setsansfont{CMU Sans Serif}
\setmainfont{CMU Serif}
\setmonofont{CMU Typewriter Text}
\usepackage{microtype}

\usepackage{hyperref}
\hypersetup{colorlinks=true, linkcolor=[RGB]{255, 3, 209}, citecolor={black}}

\usepackage{amsmath}
% \usepackage{amsthm}
% \usepackage{amssymb}

\usepackage{graphicx}

% \usepackage{caption}

\usepackage{wrapstuff}
\usepackage[ragged]{sidecap}

\usepackage[
    a4paper,
    footskip=1cm,
    headsep=0.3cm,
    top=2cm,
    bottom=2cm,
    left=2cm,
    right=2cm,
    showframe
]{geometry}
% \geometry{left=25mm,right=15mm,top=20mm,bottom=20mm}

% TODO: fonts
\pagestyle{empty}
\begin{document}
    \begin{wrapstuff}[l,top=4,type=figure, width=0.4\linewidth]
        \includegraphics[width=1\linewidth]{../Images/Knuth.jpeg}
        \caption{Дональд Ервін Кнут}
        \label{wrpstff:Knuth}
    \end{wrapstuff}
    Дональд Ервін Кнут (рис. \ref{wrpstff:Knuth}) (10 січня 1938 , Мілвокі, Вісконсин ) --- інформатик, ідеолог програмування та почесний професор Стенфордського університету.
    Автор фундаментальної праці «Мистецтво програмування»; вважається одним з батьків аналізу складності алгоритмів.
    Розробник типографічної системи \TeX{} та пов'язаної мови визначення шрифтів і системи їх рендерингу \textmd{METAFONT}.

    Кнут народився у місті Мілвокі, штат Вісконсин, в сім'ї німецьких американців Генрі Кнута та Луізи Марії Бонінг.
    Батько Дональда працював на двох роботах: викладав бухгалетерію у Старшій Школі Мілвокі та вів невелике підприємство по друку.
    Молодший Кнут, навчаючись у тій же школі, отримав багато академічних відзнак, більшість з яких за геніальні способи вирішення різноманітних проблем.
    Наприклад, у восьмому класі він взяв участь у змаганні, в якому потрібно було відшукати всі слова, які можна скласти з букв словосполуки «Ziegler's Giant Bar».
    У суддейському списку було 2500 слів, та Дональду вдалось знайти 4500 та перемогти у конкурсі.

    Освіта У 1956 році Кнут отримав запрошення до Технологічного Інституту (рис. \ref{SCfig:CWRU}) у Клівленді, Огайо, де вперше познайомився з IBM 650, одним із перших мейнфреймів.
    Прочитавши посібник до комп'ютера, Кнут вирішив переписати код компілятора для комп'ютера з його колишньої школи, тому що він вірив, що зможе зробити його краще.

    \sidecaptionvpos{figure}{c}
    \begin{SCfigure}[0.75][!h]
        \centering
        \includegraphics[width=0.5\linewidth]{../Images/CWRU.jpg}
        \caption{Західний резервний університет Кейса (CWRU)}
        \label{SCfig:CWRU}
    \end{SCfigure}

    У 1958 році Кнут створив програму, щоб допомогти шкільній баскетбольній команді вигравати більше матчів.
    Він призначив кожному гравцю «вартість», щоб оцінити імовірність кожного баскетболіста здобути очки.
    Цей підхід оцінили видання Newsweek і CBS Evening News, згадавши Кнута
    у своїх випусках.

\end{document}